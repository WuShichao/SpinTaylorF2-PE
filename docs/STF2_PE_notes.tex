\documentclass[nofootinbib,preprintnumbers,superscriptaddress,notitlepage]{revtex4-1}

\usepackage{times}
\usepackage[usenames]{color}
\usepackage{amsfonts}
\usepackage{amsmath}
\usepackage{amssymb}
\usepackage{bm}
\usepackage{dcolumn}
\usepackage{enumerate}
\usepackage{epsfig}
\usepackage{graphicx}
\usepackage{graphics}
\usepackage[latin1]{inputenc}
\usepackage{latexsym}
\usepackage{rotating}
\usepackage{hyperref}
\usepackage[caption=false]{subfig}


%setup packages
\hypersetup{
    colorlinks=true,
    linkcolor=red,
    filecolor=magenta,      
    urlcolor=magenta,
    citecolor=magenta,
}

%Define equation environment shorthand
\newcommand{\<}{\begin{equation}}
\newcommand{\?}{\end{equation}}

\begin{document}

\title{Parameter estimation using SpinTaylorF2}

\author{Soham Mukherjee}
\affiliation{Indian Institute of Science Education and Research, 
Thiruvananthapuram, India}
\author{Archana Pai}
\affiliation{Indian Institute of Science Education and Research, 
Thiruvananthapuram, India}
\date{\today}

\begin{abstract}
Investigate the faithfulness of the sidebands of SpinTaylorF2 
in parameter estimation for NS-BH binaries. Develop a consistency 
check to determine at which regions of parameter space, can the 
sidebands of SpinTaylorF2 be used instead the of full SpinTaylorF2 
waveform. This study may help determine at which regions the sidebands
are good enough for parameter estimation, and therefore reduce
computational costs associated with generating the full SpinTaylorF2 
approximant.
 
\end{abstract}

\maketitle

%%%%%%%%%%%%%%%%%
\section{Introduction}
%%%%%%%%%%%%%%%%%

See ~\cite{Lundgren2014}.

%%%%%%%%%%%%%%%%%
\section{Parameter Estimation}
%%%%%%%%%%%%%%%%%

%%%%%%%%%%%%%%%%%
\section{Results}
%%%%%%%%%%%%%%%%%


\acknowledgments

Acknowledge people here.

\bibliography{STF2_PE}

\end{document}
