%=======================================================================
% SM 4/2017
% Master's thesis Report IISER Thiruvananthapuram
% Investigations of SpinTaylorF2
%=======================================================================

\chapter{Fisher Matrix Study of SpinTaylorF2}

The standard approach used for searching for a GW signal $h(t,\bar{\mu})$ with
unknown set of parameters $\bar{\mu}$ in the data $(x)$, is to compute the
maximum of the log likelihood ratio $(\log\Lambda)$
\begin{equation}
\log{\Lambda} = \langle x\, |\, h(t,\bar{\mu})\rangle - \dfrac{1}{2}\langle h(t,\bar{\mu})\, |\, h(t,\bar{\mu})\rangle
\end{equation}
over a bank of templates $\{h(t,\bar{\mu_i})\}$, and find the point
$\bar{\mu}$ in the parameter space, where $\log{\Lambda}$ is maximum. For this
to work, the template bank should sample the parameter space \textit{finely
enough}, so that that any true signal would lie \textit{close enough} to one of the
templates in the bank. In this chapter, we first discuss what determines how
finely the parameter space needs to be sampled, or equivalently, the density
of templates required for constructing a template bank for a generic system,
and then discuss the implications of using the SpinTaylorF2 model or its
sidebands in constructing a template bank for precessing systems.

\section{Geometric approach to template bank construction}

The idea is to construct a template bank such that the mismatch between the
true signal $h(t, \bar{\lambda}) = u(\bar{\lambda})$ and the nearest template $u(t,
\bar{\lambda} + \Delta\bar{\lambda})$ in the discretized bank is below some level of
tolerance. This mismatch can be quantified in terms of an ambiguity function
$\mathcal{A}$ defined in terms of the fractional loss in the expected
SNR given by the match between the template and the true signal:
\begin{equation}
\mathcal{A} = \langle u(t,\bar{\lambda} )\,|\, u(t,
\bar{\lambda} + \Delta\bar{\lambda}) \rangle
\end{equation}
The fractional loss in SNR is given by $1-\mathcal{A}$, and therefore, for a
perfectly matched template, the fractional loss in the expected SNR would be
equal zero. Note that we assume all the templates in the template bank are
normalized, i.e., $\langle u(t,\bar{\lambda} )\,|\, u(t,\bar{\lambda}) \rangle =1$. 
Therefore, one needs to appropriately choose the parameter
mismatch $\Delta\bar{\lambda}$ between any two templates in the bank, such that
the maximum fractional SNR loss, for any given signal is below a certain
threshold. Assuming that $\Delta\bar{\lambda}$ is small, we can Taylor expand $u(t,
\bar{\lambda} + \Delta\bar{\lambda})$ and compute the ambiguity function maximized over of $\Delta\lambda$ to get
\begin{eqnarray}
\mathcal{A} &=& \langle u(\bar{\lambda})\,|\, u(\bar{\lambda}) + \Delta\lambda_i
\dfrac{\partial u}{\partial \lambda_i} (\bar{\lambda}) + \dfrac{1}{2} \Delta \lambda_i \Delta\lambda_j\dfrac{\partial^2
u}{\partial\lambda_i\lambda_j}\,(\bar{\lambda})\rangle + O\left[(\Delta\lambda)^3\right]\\
\mathcal{A}\rvert_{\Delta\lambda=0} &=& 1 + \dfrac{1}{2} \Delta \lambda_i \Delta\lambda_j \langle
u(\lambda)\,|\,\dfrac{\partial^2
u}{\partial\lambda_i\lambda_j}\,(\bar{\lambda})\rangle + O\left[(\Delta\lambda)^3\right]
\label{metric}
\end{eqnarray}
The above equation motivates the definition for a metric $g^{ij}$ using which
we can define distances $(ds^2 = 1 - \mathcal{A})$ in the parameter space as
\begin{equation}
g^{ij} = -\dfrac{1}{2}\langle u(\bar{\lambda})\,|\,\dfrac{\partial^2
u}{\partial \lambda_i \partial \lambda_j} \rangle \qquad ds^2 =
g^{ij}d\lambda_i d\lambda_j
\end{equation}
The above metric is useful for quantifying the template density required in a
particular region of parameter space. Concretely, since the volume element in
a m-dimensional manifold scales as $\sqrt{|g|}$, where $|g|$ is the
determinant of $g^{ij}$, higher the metric determinant implies template
density, for a fixed value of maximum allowed mismatch. One can re-express
$g^{ij}$, by exploiting the fact that the templates are normalized, and
therefore partial derivatives of norms of the templates are zero, as
\begin{equation}
g^{ij} = -\dfrac{1}{2}\langle \dfrac{\partial
u}{\partial\lambda_i}\,|\,\dfrac{\partial
u}{\partial\lambda_j} \rangle,
\end{equation}
and is the expression we use in our computations. Further, under the assumptions
of Gaussian stationery noise, one can show that the metric $g^{ij}$ is related to 
the Fisher information matrix~\cite{Fisher} by a constant factor of 1/2. 

\section{Fisher matrix computations for SpinTaylorF2}









