\chapter{Introduction}

The spectacular detection of gravitational waves from merging binary black
holes by Advanced LIGO~\cite{Event_0, Event_2} has firmly inaugurated the era
of gravitational wave astronomy. Future detections of merging binary black
holes (BBH), binary neutron stars (BNS), or neutron star-black hole (NSBH)
binaries by Advanced LIGO and other upcoming  gravitational wave
observatories~\cite{KAGRA, Virgo, LIGO_india} would enable us to probe the
strong-field, dynamical regime of gravity in much greater detail, allowing for
more stringent tests of the theory of general relativity ~\cite{Berti2015}, and
would help in understanding the nature of matter at supra-molecular densities,
for example, in the core of neutron stars~\cite{Agathos,Chatziioannou}. In
addition, joint observations in the electromagnetic and gravitational wave band
is expected to open up a new domain of multi-messenger astronomy that could
provide crucial insights into the progenitors of highly energy events such as
short gamma-ray bursts~\cite{Arun2014} and Type 1a or core-collapse
supernovae~\cite{Falta2011,Ott2013}.

Gravitational waves (GWs) are solutions to the vacuum Einstein's field
equations, generated by sources with a time-varying quadrupole
moment~\cite{Creighton}. These waves couple very weakly with matter, which
make them astrophysically very interesting, but also make them extremely
difficult to detect. They carry away energy, linear and angular momentum from
the system, and therefore, a compact binary radiating GWs, would eventually
coalesce~\cite{Peters1963}. These waves encode information about the
astrophysical source that generates them. In particular, if one or both of the
spin angular momenta of the components is misaligned with the total angular
momentum of the binary, the orbit is expected to
precess~\cite{Apostolatos1994}, which introduces rich structure in the
inspiral phase of the waveform; see~Fig.~\ref{fig:waveforms}. Observing and
analyzing precessing systems, therefore, would enable us to investigate the
spin-orientations of the components, that could provide astrophysically
significant information about their formation
channel~\cite{AstrophysicalImplications,Rodriguez}, as well as the interesting
dynamics of these systems.

Probing the precession in GW signals, however, is a non-trivial problem---it
scales in theoretical complexity and computational cost as more physical
effects such as spin-induced precession is~\cite{Apostolatos1994} is added to
the waveform models used for searching for the signal. For precessing systems,
the problem is two fold: first, the cost of generating precessing waveforms is
much higher than their non-precessing counterparts, and second, precessing
systems require a higher dimensional, highly dense template bank, which leads
to a major increase in the overall computational cost~\cite{PE_cost, Nat2017}.
At the same time, incorporating the effects of precession is crucial for
detection as well as parameter estimation, to avoid SNR loss when filtering
the signal with the template bank, which could lead to false dismissals of
real events, or even bias in the estimated parameters of the
signal~\cite{Bias_1}. Spin-induced precession is a generic feature of compact
binaries.  Therefore, there is a need to incorporate the effects of precession
in a waveform model in a computationally efficient manner.

SpinTaylorF2 ~\cite{Lundgren2014} is a computationally efficient
and analytically-tractable, frequency waveform model for GWs emitted during
the inspiral phase from single-spin precessing binaries, such as NSBH systems.
For NSBH systems, the black hole (BH) spin can, in principle, reach the limit
predicted by GR and can have any generic orientation. However, indirect
observations suggest that the spin of the companion NS would be much smaller.
While isolated NS are known to have dimensionless spin up to $\sim 0.4$ in the
case of PSR~J1748-2446ad~\cite{PSR2006}, the spin of NS in binary systems is
observered to be much lesse: the fastest spinning pulsar in a double NS
binary (PSR~J0737-3039A) has a spin $\sim 0.05$~\cite{BURGAY2003}. Hence, for
all practical purposes, the spin of NS can be neglected \cite{Lorimer:2008se}.

The SpinTaylorF2 waveform can be expressed as a sum of five sidebands (also
referred to as spin-harmonics) and each sideband corresponds to a modulation
of a leading order function. The relative amplitude of the sidebands depend on
the extent of precession of the system. One can therefore ask whether it's
possible to use a single (or a combination of) sideband(s) for probing the
precession of the system. In this project, we determine if such an approach is
indeed possible, and the regions of parameter space where the approach would
be applicable. Further, we use the information on the relative contribution
different sidebands to the full waveform to put bounds on one of the spin
parameters of the source. In addition, we perform a Fisher matrix analysis for
a specific set of sidebands and investigate the possibility of constructing a
template bank for precessing systems using indivial harmonics of the
SpinTayloF2 model.
