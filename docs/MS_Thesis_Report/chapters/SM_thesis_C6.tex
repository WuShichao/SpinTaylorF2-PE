%=======================================================================
% SM 4/2017
% Master's thesis Report IISER Thiruvananthapuram
% Investigations of SpinTaylorF2
%=======================================================================

\chapter{Discussions and Future Directions}

In this thesis, we focused our study on the single spin, precessing
gravitational waveform model SpinTaylorF2, and explored numerically, the
consequences of the mathematical structure of the waveform, in terms of the
fractional SNR contribution of the individual sidebands to the total SNR of
the full waveform model. We showed that precession has observable consequences
on the total SNR pattern in the spin-precession parameter space. Specifically,
we investigated how the observed patterns vary over the mass-spin parameter
space, a task that is not analytically tractable. Our results, however, are in
agreement with what one would predict from the analytical form of the
expressions for precessing waveforms~\cite{Apostolatos1994}.

Further, we explored how the total SNR is distributed among the various
sidebands of the waveform model. We discovered that only two of the sidebands
($m=2$ and $m=0$) are dominant throughout the entire parameter space, and show
strong complementarity in their regions of dominance: the $m=0$ mode is
dominant for strongly precessing systems, whereas $m=2$ dominates when the
extent of precession in the system is low or non-existent. This complementary
nature of the two sidebands, allowed us to partition the spin-precession
parameter space into 3 regions (strongly precessing, moderately precessing,
and mildly precessing systems), based on the overlap difference between the
two sidebands. We obtained the numerical expressions for the boundaries of
these regions, which, for strongly and moderately precessing systems allowed
us to put an upper bound on the value of the spin-alignment parameter
$\kappa$. Most of this work is a part of a manuscript in
preparation~\cite{STF2}.

However, a few open questions remain, which require further attention. For
example, we are unsure about the origin of a set of discontinuous points (see
last panel in Fig.~\ref{fig:P2}~b) in the $(\theta_J,
\kappa)$ space, and the `beats' we observe in the overlap
(Fig.~\ref{fig:P2}~b) and SNR (Fig.~\ref{fig:SNR}) figures. We expect the
latter to be due to the interference terms in the overlap computation between
the sidebands, but further investigations are required on this front. One
would also to like generalize our numerical fits to any arbitrary $\psi_J$
(currently we use $\psi_J = 0.001$ for all our analyses) and test their
robustness against different detector noise realizations. It would also be
very interesting explore whether one can put a bound on the BH spin $\chi_1$
using the existing numerical data, similar to what has been for $\kappa$.

A particularly promising avenue of investigation, which we are currently
working on, is to carry about a Fisher matrix study~\cite{Fisher}, followed by
a fitting factor analysis, to investigate the feasibility and efficacy of
constructing a template bank for precessing systems using only the $m=0$ mode
of the waveform. To our knowledge, currently no robust and efficient method to
construct a template bank for generic precessing systems exists in literature.
A recent study~\cite{Nat2017}, investigated this problem and used the
SpinTaylorF2 waveform model to construct a template bank for face-on and 
face-off systems undergoing precession. However, for the edge-on cases, they found
that the number of templates required increase exponentially, which
significantly pushes up the computation cost. Our preliminary results suggest
that using a single sideband ($m=0$) instead of the full waveform to construct
the template bank, may help ameliorate the issue.

A major part of this project involved elaborate code development to
efficiently compute the SNR and the overlaps over multiple dimensions of
parameter space, and our codes relied heavily on the existing GW data analysis
infrastructure provided by the PyCBC~\cite{Canton:2014ena,Usman:2015kfa}
library and the LALSuite software package. The codes used in this project are
available upon request.



