\chapter{Introduction}

The spectacular detection of gravitational waves from merging binary black holes
by Advanced LIGO~\cite{Event_0, Event_2} has firmly inaugurated the era of
gravitational wave astronomy. Future detections of merging binary neutron stars
(BNS) or neutron star-black hole (NS-BH) binaries by Advanced LIGO and other
upcoming  gravitational wave observatories~\cite{KAGRA, Virgo, LIGO_india} would
allow us to probe the strong-field, dynamical regime of gravity in much greater
detail.

Gravitational waves encode information about the astrophysical sources that
generate them. One of the key goals of gravitational  wave astronomy is to
faithfully estimate the astrophysical parameters --- masses and spins of the
components of the binary, source location in the sky, and the equation of state
if the signal if the source comprises of neutron stars --- encoded in the
gravitational wave signal. Parameter estimation is a non-trivial  problem; it
scales both in theoretical complexity and comptuational cost as more physical
details such as spin-induced precession and tidal effects are added to the
waveform model. At the same time, accurate waveform models are crucial if one
wants to faithfully recover the source parameters; unmodelled parameters in the
waveform model leads to systematic biases in the estimation~\cite{Bias_1,
Bias_2}.

Spin-induced precession~\cite{Apostolatos1994} is a generic feature of compact
binaries.  However, including the effects of precession in a waveform model
significantly increases the computational cost associated with waveform
generation~\cite{PE_cost}, and therefore, parameter estimation studies usually
neglect the effects of precession. However, studies that neglect precession are
expected to suffer from systematic bias in the estimated parameters;
see~\cite{Bias_1, Bias_3}  for a discussion. Therefore, there is a need to
incorporate the effects of precession in a waveform model in a computationally
efficient manner.

The SpinTaylorF2 waveform~\cite{Lundgren2014} is a computationally efficent and
analytically-tractable, frequency waveform model for gravitational waves from
generic precessing binaries that is applicable to a single-spin binary. The
model therefore, can be applied NS-BH binaries since the neutron star in a
binary is expected to have negligible spin~\cite{NSBH_upperlims, Brown2012,
Kramer}. The waveform can be expressed as a sum of five terms (referred to as
sidebands) and each sideband corresponds to a modulation of a leading order
function. The relative ampltiude of the sidebands depend on the extent of
precession of the system: for a non-precessing signal,  only the one sideband
($m=2$) survives i.e. has a non-zero amplitude, but when the system is
precessing, all of the sidebands develop a non-zero amplitude, and therefore
start contributing to the amplitude of the total waveform. One can therfore ask
whether it's possible to use a single (or a combination of) sideband(s) for
parameter estimation, instead of the full SpinTaylorF2 waveform, which would
result in  a massive reduction of comptutational cost.

In this project, we determine if such an approach is indeed possible, and the
regions of  parameter space where the approach would be applicable. We do this
by computing the signal-to-noise ratio (SNR) of the full SpinTaylorF2 waveform,
and then investigating the fractional SNR contribution of each of the sidebands
to the total SNR, by computing the overlap~\cite{Creighton} of  the sideband
with the full waveform, in different regions of the parameter space. We observe
that the ($m=2$) and ($m=0$) sidebands are the dominant contributors to the
total SNR in roughly complimentary regions in the $(\theta_J, \kappa)$ parameter
space (see Sec.~(\ref{chap:SpinTaylorF2}) for definitions) --- the
ratio of  SNR due to the ($m=0$) sideband to the SNR corresponding to the
($m=2$) sideband is  greater than one if the system is highly precessing, less
than one for  mildly precessing cases, thus indicating the suitability of using
a particular sideband for parameter estimation in different regions of the
parameter space.

In the following sections, we define our notations and briefly discuss the
construction  of the SpinTaylorF2 waveform. We then present our results of the
SNR and overlap computations, and discuss the possibilities of using these
results in parameter estimation studies.

