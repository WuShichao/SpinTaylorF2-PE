%=======================================================================
% SM 4/2017
% Master's thesis Report IISER Thiruvananthapuram
% Investigations of SpinTaylorF2
%=======================================================================

\chapter{The Stationary Phase Approximation}

The stationary phase approximation is commonly used to compute the Fourier
transforms of oscillatory signals. For a time-domain GW signal of the form
\begin{equation}
h_{+}(t) = A(t_{\rm ret}) \cos\Phi(t_{\rm ret}), 
\end{equation}
where $t_{\rm ret}$ is the retarded time at coalescence, the Fourier 
transform of the signal would be given by
\begin{eqnarray}
\tilde{h}_{+}(f) &=& \int dt A(t_{\rm ret}) \cos\Phi(t_{\rm ret}) e^{2\pi i f t},\\
 &=& \dfrac{1}{2} e^{2\pi i f r/c} \int dt_{\rm ret}\, A(t_{\rm ret}) \left[
 e^{i \Phi(t_{\rm ret})} + e^{-i \Phi(t_{\rm ret})} \right] e^{2\pi i f t_{\rm
 ret}} \nonumber
 \label{td}
\end{eqnarray}
where we rename the integration variable from $t$ to $t_{\rm ret}$. We can
compute this integral using the Stationary Phase Approximation (SPA). The main
idea behind the approximation is that the contribution to the above integral,
provided the amplitude varies slowly, would come only from points where the
phase is stationary; in the rest of the regions, the oscillatory nature of the
phase would cancel out to give a negligible net contribution to the integral.

We note that the phase term proportional to $e^{i \Phi(t) + 2 i \pi f
t}$ has no stationary points, and hence has no effect on the value of the
integral. However, $e^{-i \Phi(t) + 2 i \pi f t}$, has a stationary
point $t_s (f)$ given by the condition $2 \pi f = \dot{\Phi}(t_s)$. Hence, we 
focus on the following integral
\begin{equation}
h_{+}(t) = \dfrac{1}{2} e^{2\pi i f r/c} \int dt_{\rm ret}\, A(t_{\rm ret})
 \left[e^{-i \Phi(t) + 2\pi i f t} \right]  \nonumber,
\end{equation}


