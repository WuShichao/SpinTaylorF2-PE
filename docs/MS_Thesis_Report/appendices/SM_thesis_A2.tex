%=======================================================================
% SM 4/2017
% Master's thesis Report IISER Thiruvananthapuram
% Investigations of SpinTaylorF2
%=======================================================================

\chapter{The Stationary Phase Approximation}

The stationary phase approximation is commonly used to compute the Fourier
transforms of oscillatory signals. For a time-domain GW signal of the form
\begin{equation}
h_{+}(t) = A(t_{\rm ret}) \cos\Phi(t_{\rm ret}), 
\end{equation}
where $t_{\rm ret}$ is the retarded time at coalescence, the Fourier 
transform of the signal would be given by
\begin{eqnarray}
\tilde{h}_{+}(f) &=& \int dt A(t_{\rm ret}) \cos\Phi(t_{\rm ret}) e^{2\pi i f t},\\
 &=& \dfrac{1}{2} e^{2\pi i f r/c} \int dt_{\rm ret}\, A(t_{\rm ret}) \left[
 e^{i \Phi(t_{\rm ret})} + e^{-i \Phi(t_{\rm ret})} \right] e^{2\pi i f t_{\rm
 ret}} \nonumber
 \label{td}
\end{eqnarray}
where we rename the integration variable from $t$ to $t_{\rm ret}$. We can
compute this integral using the Stationary Phase Approximation (SPA). The main
idea behind the approximation is that the contribution to the above integral,
provided the amplitude varies slowly, would come only from points where the
phase is stationary; in the rest of the regions, the oscillatory nature of the
phase would cancel out to give a negligible net contribution to the integral.

We note that the phase term proportional to $e^{i \Phi(t) + 2 i \pi f
t}$ has no stationary points, and hence has no effect on the value of the
integral. However, $e^{-i \Phi(t) + 2 i \pi f t}$, has a stationary
point $t_s (f)$ given by the condition $2 \pi f = \dot{\Phi}(t_s)$. Hence, we 
focus on the following integral
\begin{equation}
\tilde{h}_{+}(f) = \dfrac{1}{2} e^{2\pi i f r/c} \int dt_{\rm ret}\, A(t_{\rm ret})
 \left[e^{-i \Phi(t) + 2\pi i f t} \right]  \nonumber,
\end{equation}
and Taylor expand the phase factor $(-i \Phi(t) + 2\pi i f t)$ around the
stationary point $t_s$, which, after a change of variables gives
\begin{equation}
\tilde{h}_{+}(f) = \dfrac{1}{2} e^{2\pi i f r/c} A(t_{s})\,
e^{i  \left[-\Phi(t_s) + 2\pi i f t_s\right]} 
\sqrt{\dfrac{2}{\ddot{\Phi}(t_s)}}\int dx\, e^{-i x^2}.
\end{equation}
The last integral is a Fresnel integral
\begin{equation}
\int_{-\infty}^{\infty} dx\, e^{i x^2} =  \sqrt{\pi} e^{-i\pi/4}.
\end{equation}
Using the above result, we arrive at the final expression for the waveform in
the frequency domain
\begin{equation}
\tilde{h}_{+}(f) = \dfrac{1}{2} A(t_{s})\,\sqrt{\dfrac{2}{\ddot{\Phi}(t_s)}}\,
e^{i\left[2\pi f(t_s - r/c) - \Phi(t_s) - \pi/4 \right]}.
\end{equation}
For the SpinTaylorF2 waveform, we start with the time domain expression of the
form
\begin{equation}
h_{+}(t) = \dfrac{2 M \eta}{D}v^2 \text{Re} \left[z(\alpha - \phi, \theta,
\psi, \beta) e^{2 i (\Phi - \zeta)}\right]
\end{equation}
Upon subsituting the expression for $z_m$'s, the total phase is given by the
combination $\Phi_m = 2 (\Phi - \zeta) + m \alpha$. At the level of
approximation employed here, we assume that the factor $e^{i(-2\zeta +
m\alpha)}$ is a slowly-varying term, and therefore, is pulled out of the
integration. The resulting phase, after the Fourier transform, is given by
$\Psi(f) = 2\pi f t(f) - 2\Phi(t(f))$. By approximating the stationary phase
amplitude $1/\sqrt{\ddot\Phi}$ by the leading order term proportionaly to
$f^{-7/6}$, we finally arrive at the expression for the waveform model in the
frequency domain

\begin{equation}
\tilde{h}_{+}(f) = \dfrac{2\pi M_{c}^{2}}{D}\sqrt{\dfrac{5}{96\pi}}(\pi M
f)^{-7/6}\sum_{m}z_{m}e^{i(\Psi(f) - 2\zeta(f)) + i m \alpha(f)}
\end{equation}