%=======================================================================
% SM 4/2017
% Master's thesis Report IISER Thiruvananthapuram
% Investigations of SpinTaylorF2
%=======================================================================

\chapter{Precession amplitude functions $z_m$}

\label{append}
The normalized amplitude factor $z_m$ can be expressed in terms of 
spin-weighted spherical harmonics $\mathstrut_{-2}Y_{2,m}$. When expanded, the
simplified expressions of $z_m$'s  as functions of $(\theta_J,
\psi_J, \beta(f))$ can be written as

\begin{eqnarray}
z_2 &=& \cos^4 \frac{\beta}{2}\left[e^{-i2\psi_J}~\cos^4 \frac{\theta_J}{2}+e^{i2\psi_J}~\sin^4 \frac{\theta_J}{2} \right]~,\\
z_1 &=&\sin\frac{\beta}{2}~\cos^3\frac{\beta}{2}~\sin \theta_J \left[e^{-i2\psi_J} (1 +  \cos \theta_J) + e^{i2\psi_J} (1- \cos \theta_J)  \right]  ~,\\
z_0 &=& \frac{3}{4} \sin^2 \beta~\cos2 \psi_J~\sin^2 \theta_J~\\
z_{-1} &=& \cos\frac{\beta}{2}~\sin^3\frac{\beta}{2}~\sin \theta_J \left[e^{-i2\psi_J} (1 -  \cos \theta_J) + e^{i2\psi_J} (1+ \cos \theta_J)  \right]  ~,\\
z_{-2} &=& \sin^4\frac{\beta}{2}~\left[e^{-i2\psi_J} \sin^4\frac{\theta_J}{2} + e^{i2\psi_J} \cos^4\frac{\theta_J}{2} \right] ~.
\end{eqnarray}

For non-precessing systems, $\beta = 0$, since the orbital angular momentum
$\mathbf{L}$ is aligned with the total angular momentum $\mathbf{J}$. Note
that, except the expression for $z_2$, all other $z_m$'s are functions of
$\sin{\beta}$. Therefore, for $\beta = 0$, all other amplitudes, expect for
the $z_2$ mode, go to zero. Therefore, $m=2$ mode dominates for non-precessing
cases.
