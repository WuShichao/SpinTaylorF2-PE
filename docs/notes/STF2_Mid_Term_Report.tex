\documentclass[nofootinbib,preprintnumbers,superscriptaddress,notitlepage]{revtex4-1}

\usepackage{times}
\usepackage[usenames]{color}
\usepackage{amsfonts}
\usepackage{amsmath}
\usepackage{amssymb}
\usepackage{bm}
\usepackage{dcolumn}
\usepackage{enumerate}
\usepackage{epsfig}
\usepackage{graphicx}
\usepackage{graphics}
\usepackage[latin1]{inputenc}
\usepackage{latexsym}
\usepackage{rotating}
\usepackage{hyperref}
\usepackage[caption=false]{subfig}


%setup packages
\hypersetup{
    colorlinks=true,
    linkcolor=red,
    filecolor=magenta,      
    urlcolor=magenta,
    citecolor=magenta,
}

%Define equation environment shorthand
\newcommand{\<}{\begin{equation}}
\newcommand{\?}{\end{equation}}

\begin{document}

\title{Overlap analysis of sidebands of SpinTaylorF2}

\author{Soham Mukherjee}
\affiliation{Indian Institute of Science Education and Research, 
Thiruvananthapuram, India}
\date{\today}

\begin{abstract}
We study the SpinTaylorF2 waveform, a closed-form, single spin, frequency 
domain waveform. The the dominant (2,2) mode can be further decomposed 
into sidebands. The SpinTaylorF2 waveform is most suited for neutron star 
black hole binaries, since in such systems the neutron star is expected to 
have negligible spin compared to that of the black hole. In thus study, we 
compare the contribution to the total SNR from the sidebands. We find that 
(2,2,2) and (2,2,0) are dominant in complimentary regions of the parameter 
space -- (2,2,2) mode dominates when there is no precession and (2,2,0) is 
seen to dominate when the system is highly precessing. This would allow us 
to use a particular sideband of the SpinTaylorF2 waveform, instead of the
full SpinTaylorF2 (which more significantly more computationally expensive) 
for parameter estimation. 
\end{abstract}

\maketitle

%%%%%%%%%%%%%%%%%
\section{Introduction}
%%%%%%%%%%%%%%%%%
With the discovery of GW150194, gravitational wave astronomy has ushered into
a new era. Among the most likely candidates that is expected to be detected by
LIGO, neutron star-black hole binaries are yet to be discovered. In fact we 
don't know if they even exist, and thus it would be very interesting if LIGO
is able to detect and discern if the observed signal is from a NS-BH binary. 
Such systems can be formed in dynamical capture, which may lead to orbits with 
high eccentricity, or can be formed in the common envelope phase where one star 
collapses into a black hole before the other. However, in both cases, the spin 
of the neutron star is expected to be significantly smaller than the black hole,
primarily for two reasons: neutron stars can only spin as fast as the rotational 
breakup frequency and black holes can be spun up with accretion. 

To detect such systems, LIGO current uses a matched filter approach, which 
necessitates the use of highly accurate waveforms. 

See ~\cite{Lundgren2014}.

%%%%%%%%%%%%%%%%%
\section{Parameter Estimation}
%%%%%%%%%%%%%%%%%

%%%%%%%%%%%%%%%%%
\section{Results}
%%%%%%%%%%%%%%%%%


\acknowledgments

Acknowledge people here.

\bibliography{STF2_PE}

\end{document}
